\documentclass[10pt]{article}

% Convert PDF to PNG:
% $ pdftoppm flowchart.pdf outputname -png -rx 300 -ry 300

\usepackage{tikz} % este paquete y el de abajo se encargan de crear los
%'flow charts'
\usetikzlibrary{matrix,shapes,arrows,positioning,chains}
% Add shadows to tikz boxes.
\usepackage{pstricks}
\usetikzlibrary{shadows.blur}
\usetikzlibrary{shapes.symbols}

% No page numbering.
\pagestyle{empty}

\begin{document}

%Define block styles
\tikzset{
    decision/.style={
        diamond,
        draw,
        text width=5em,
        text badly centered,
        inner sep=0pt
    },
    block/.style={
        rectangle,
        draw,
        text width=14em,
        text centered,
        rounded corners
    },
    block2/.style={
        rectangle,
        draw,
        text width=7em,
        text centered,
        rounded corners
    },
    cloud/.style={
        draw,
        ellipse,
        text width=5em,
        minimum height=2em
    },
    cloud2/.style={
        draw,
        ellipse,
        text width=6em,
        minimum height=2em
    },
    descr/.style={
        fill=white,
        inner sep=2.5pt
    },
    connector/.style={
     -latex,
     font=\normalsize
    },
    rectangle connector/.style={
        connector,
        to path={(\tikztostart) -- ++(#1,0pt) \tikztonodes |- (\tikztotarget) },
        pos=0.5
    },
    rectangle connector/.default=-2cm,
    straight connector/.style={
        connector,
        to path=--(\tikztotarget) \tikztonodes
    }
}

\begin{tikzpicture}
\matrix (m)[matrix of nodes, column sep=1.5cm, row sep=0.75cm, align=center,
nodes={rectangle,draw, anchor=center}, ampersand
replacement=\& ]
{
|[cloud, top color=green!25, bottom color=green!15, blur shadow={shadow blur
steps=5}]| {Start} \\
|[block, top color=yellow!25, bottom color=yellow!15,blur shadow={shadow blur
steps=2}]| {Apply PCA and \\features reduction on data}    \\
|[block, top color=blue!25, bottom color=blue!15,
blur shadow={shadow blur steps=2}]|{Run the selected\\clustering algorithm}\\
|[block, top color=blue!25, bottom color=blue!15, blur shadow={shadow blur
steps=2}]| {Reject clusters consistent with a random uniform distribution}\\
|[block, top color=blue!25, bottom color=blue!15,
blur shadow={shadow blur steps=2}]| {Apply GUMM cleaning}\\
|[decision, top color=yellow!25, bottom color=yellow!15, blur shadow={shadow
blur steps=2}]| {Are all the ``fake clusters'' rejected?}   \\
|[block, top color=yellow!25, bottom color=yellow!15,
blur shadow={shadow blur steps=2}]|
{Identify surviving stars as star cluster members and all others
as field stars}\\
|[block, top color=blue!25, bottom color=blue!15, blur shadow={shadow blur
steps=2}]| {Estimate KDE probabilities for members}\\
|[decision, top color=yellow!25, bottom color=yellow!15, blur shadow={shadow blur steps=2}]| {Is the Outer Loop finished?}\\
|[cloud, top color=red!25, bottom color=red!15, blur shadow={shadow blur
steps=2}]| {End}   \\
};
\path [draw=black,solid,line width=0.4mm,>=latex,->] (m-1-1) edge (m-2-1);
\path [draw=black,solid,line width=0.4mm,>=latex,->] (m-2-1) edge (m-3-1);
\path [draw=black,solid,line width=0.4mm,>=latex,->] (m-3-1) edge (m-4-1);
\path [draw=black,solid,line width=0.4mm,>=latex,->] (m-4-1) edge (m-5-1);
\path [draw=black,solid,line width=0.4mm,>=latex,->] (m-5-1) edge (m-6-1);
\path [draw=black,solid,line width=0.4mm,>=latex,->] (m-6-1) edge (m-7-1);
\draw [draw=black,solid,line width=0.4mm,>=latex,->] (m-6-1.east) -- ++(2,0) |-  (m-3-1.east)
node[xshift=-1cm][pos=0,above,text=black]{\textbf{No}}
node[near start,sloped,above,rotate=180] {\textbf{Inner Loop}};
\path [draw=black,solid,line width=0.4mm,>=latex,->] (m-7-1) edge (m-8-1)
node[xshift=0.6cm][yshift=0.85cm][above,text=black]{\textbf{Yes}};
\path [draw=black,solid,line width=0.4mm,>=latex,->] (m-8-1) edge (m-9-1);
\draw [draw=black,solid,line width=0.4mm,>=latex,->] (m-9-1.west) -- ++(-2,0) |-  (m-2-1.west)
node[xshift=1cm][pos=0,above,text=black]{\textbf{No}}
node[near start,sloped,above] {\textbf{Outer Loop}};
\path [draw=black,solid,line width=0.4mm,>=latex,->] (m-9-1) edge (m-10-1)
node[xshift=0.6cm][yshift=-2.15cm][above,text=black]{\textbf{Yes}};
\path [draw=black,solid,line width=0.4mm,>=latex,->] (m-9-1) edge (m-10-1);
\end{tikzpicture}
\end{document}
